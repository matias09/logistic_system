\documentclass[
10pt, % Main document font size
a4paper, % Paper type, use 'letterpaper' for US Letter paper
oneside, % One page layout (no page indentation)
%twoside, % Two page layout (page indentation for binding and different headers)
headinclude,footinclude, % Extra spacing for the header and footer
BCOR5mm, % Binding correction
]{scrartcl}

\title{Tp Sistema de Logistica}
\author{Matias M. Marceca}
\date{}

\begin{document}

\maketitle

\tableofcontents

\pagebreak

\section{Idea y Oportunidad de Negocio}

\subsection{Que tiene de innovadora nuestra idea ?}
\textbf{La idea tiene de innovacion :}

\begin{itemize}
    \item { La creacion de un sistema modular. }
    \item { La posibilidad de sincronizar tareas entre distintas areas
        tales como ( proveedores - transportistas - clientes - distribuidoras )
        todo por medio de una aplicacion mobile sincronizada con
        servidores en la nube.}
\end{itemize}

\subsection{Cuales son las necesidades o deseos de los consumidores
            que van a satisfacer ?}
\textbf{Las necesidades a satisfacer son las siguientes :}

\begin{itemize}
    \item { Mayor organizacion a la hora de gestionar envios de mercaderias
              desde la salida del Deposito hasta la puerta del comprador/es. }
    \item { Mejor trazabilidad de rutas a destinos. }
    \item { Verificacion de entregas por medio de aplicacion mobile
        desde el punto de entrega del transportista. }
\end{itemize}

\subsection{Cuales son los beneficios que nuestro producto (bien o servicio)
             ofrecera al cliente ?}
\textbf { Los beneficios que el producto ofrece son: }

\begin{itemize}
    \item { Mayor tiempo de busqueda en nuevas oportunidades de negocio. }
    \item { Tranquilidad y Visibilidad en cada etapa del proceso de entrega
              de productos. }
    \item { Sistema de valoracion de entregas por medio de aplicacion mobile
        desde el cliente. }
\end{itemize}

\subsection{Porque el cliente comprara nuestro producto ?}
\textbf {El cliente compraria nuestro producto por :}

\begin{itemize}
    \item { Muchas variedades de soluciones para distintas areas del proceso
              de transporte de productos. }
    \item { Alta confiabilidad y seguridad de datos. }
\end{itemize}

\subsection{Puede este negocio generar dinero ?}
  Definitivamente.

  La estrategia de venta va a ser basada en
  distintos Packs de soluciones por medio de modulos
  a demanda del cliente.

  Todas la nuevas caracteristicas que el cliente quiera agregar
  al producto base no perjudicara el trabajo actual ya realizado
  por el cliente.

  Tambien se puede ofrecer Soporte en horarios de trabajo.

\section {Viabilidad de Negocio}

\subsection{Proceso Productivo}
\begin{itemize}
  \item Materiales Necesarios
  \begin{itemize}
    \item Discos Externos
    \item Pendrives
    \item Ambiente no mas de 20 Grados Cent.
  \end{itemize}

  \item Tecnologia Necesaria
  \begin{itemize}
    \item Internet, con velocidades no menores a
            1GB de bajada y 500MB de subida.
    \item Telefonia
    \item Luz
  \end{itemize}

  \item Mano de Obra
  \begin{itemize}
    \item No mas de 5 desarrolladores Ssr.
  \end{itemize}

  \item Insumos
  \begin{itemize}
    \item Monitores
    \item Teclados
    \item Mouse
    \item Gabinetes armados con lo mejor en hardware al alcance disponible
    \item Cables de Red
    \item Modem o Router Wifi
    \item Expansores de Señal de Wifi
  \end{itemize}
\end{itemize}

\subsection{Localizacion}
\begin{itemize}
  \item Costo de transporte
  \begin{itemize}
    \item No habria costo, porque cada uno podria trabajar
            desde su domicilio, siempre y cuando pueda cumplir
            las condiciones de los Materiales Necesarios.
  \end{itemize}

  \item Comunicaciones
  \begin{itemize}
    \item Todo tipo de comunicacion se realizaria a traves de internet
            compartiendo codigo del proyecto en un repositorio privado.
  \end{itemize}
\end{itemize}


\subsection{Requerimientos Tecnicos}
\begin{itemize}
  \item Tecnologias
  \begin{itemize}
    \item Librerias de libre uso y distribucion para el manejo de ventanas
          y herramientas varias (Botones ejecutables - Barras de desplazamiento )
          en el sistema operativo objetivo.
  \end{itemize}

  \item Estructura Organizacional
  \begin{itemize}
    \item Lider de Proyecto
    \item Administrador de Proyecto
    \item Desarrolladores
  \end{itemize}

  \item Mapa de Proceso
  \begin{itemize}
    \item Entrevista con el Cliente
    \item Relevamiento de Informacion a tener en cuenta
    \item Preparacion de Demostrativo
    \item Se plantean 2 semanas de continuas entregas de
            pequeñas caracteristicas funcionales a evaluacion
            del cliente para una mayor frecuencia de respuesta
            sobre el producto.

    \item Preparacion de etapa Pre-Produccion. Para ello se genera un ambiente
            en servidores maqueta para simular principales
            comportamientos del sistema.

    \item Preparacion de Puesta a Produccion del sistema en servidores del cliente
  \end{itemize}
\end{itemize}


\end{document}
