\documentclass[
10pt, % Main document font size
a4paper, % Paper type, use 'letterpaper' for US Letter paper
oneside, % One page layout (no page indentation)
%twoside, % Two page layout (page indentation for binding and different headers)
headinclude,footinclude, % Extra spacing for the header and footer
BCOR5mm, % Binding correction
]{scrartcl}

\title{Tp Sistema de Logistica}
\author{Matias M. Marceca}
\date{}

\begin{document}

\maketitle

\tableofcontents

\pagebreak

\section{Idea y Oportunidad de Negocio}

\subsection{Que tiene de innovadora nuestra idea ?}
\textbf{La idea tiene de innovacion :}

\begin{itemize}
    \item { La creacion de un sistema modular. }
    \item { La posibilidad de sincronizar tareas entre distintas areas
        tales como ( proveedores - transportistas - clientes - distribuidoras )
        todo por medio de una aplicacion mobile sincronizada con
        servidores en la nube.}
\end{itemize}

\subsection{Cuales son las necesidades o deseos de los consumidores
            que van a satisfacer ?}
\textbf{Las necesidades a satisfacer son las siguientes :}

\begin{itemize}
    \item { Mayor organizacion a la hora de gestionar envios de mercaderias
              desde la salida del Deposito hasta la puerta del comprador/es. }
    \item { Mejor trazabilidad de rutas a destinos. }
    \item { Verificacion de entregas por medio de aplicacion mobile
        desde el punto de entrega del transportista. }
\end{itemize}

\subsection{Cuales son los beneficios que nuestro producto (bien o servicio)
             ofrecera al cliente ?}
\textbf { Los beneficios que el producto ofrece son: }

\begin{itemize}
    \item { Mayor tiempo de busqueda en nuevas oportunidades de negocio. }
    \item { Tranquilidad y Visibilidad en cada etapa del proceso de entrega
              de productos. }
    \item { Sistema de valoracion de entregas por medio de aplicacion mobile
        desde el cliente. }
\end{itemize}

\subsection{Porque el cliente comprara nuestro producto ?}
\textbf {El cliente compraria nuestro producto por :}

\begin{itemize}
    \item { Muchas variedades de soluciones para distintas areas del proceso
              de transporte de productos. }
    \item { Alta confiabilidad y seguridad de datos. }
\end{itemize}

\subsection{Puede este negocio generar dinero ?}
  Definitivamente.

  La estrategia de venta va a ser basada en
  distintos Packs de soluciones por medio de modulos
  a demanda del cliente.

  Todas la nuevas caracteristicas que el cliente quiera agregar
  al producto base no perjudicara el trabajo actual ya realizado
  por el cliente.

  Tambien se puede ofrecer Soporte en horarios de trabajo.

\pagebreak

\section {Viabilidad de Negocio}

\subsection{Proceso Productivo}
\begin{itemize}
  \item \textbf {Materiales Necesarios}
  \begin{itemize}
    \item Se requiere el uso de discos externos para la administracion de
            guardado de informacion relevante al desarrollo del producto.
    \item Se requiere el uso de pendrives para el facil y comodo traslado
            de archivos de video, librerias y graficos para ser utilizados
            el dia que se requiera una presentacion al cliente en las oficinas
            del mismo.
    \item Se requiere utilizar un ambiente de oficina  contando con no mas
            de 20 Grados Cent para una mayor refrigeracion y rendimiento de los
            equipos tecnicos utilizados para lleva a cabo las tareas de desarrollo.
  \end{itemize}

  \item \textbf {Tecnologia Necesaria}
  \begin{itemize}
    \item Se requiere una conexion a Internet, con velocidades no menores a
            1GB de bajada y 500MB de subida para mantener comunicacion
            constante, rapida, confiable e ininterrumpida entre diferentes
            miembros de la empresa. Esto habilita estar a la vanguardia de las
            ultimas actualizaciones en desarrollo y la utilizacion de
            herramientas en Internet para una mayor administracion de los
            procesos de tareas y estado de los desarrollos.
    \item Se requiere la utilizacion de telefonia celular entre los integrantes
            de los equipos para mantener contacto en posibles escenarios donde
            se deba realizar operaciones de soporte fuera de los horarios de
            oficina para el mantenimiento en servidores donde se encuentran
            alojadas los desarrollos realizados y estado de pruebas.

    \item Se requiere de un buen servicio de luz, ya que sin el nada seria
            posible. Como estrategia en caso de emergencias ( un corte de luz ),
            se utilizaria un generador electrico para suministrar los
            equipos electronicos, ya que varios de ellos requieren mantenerse
            prendidos en todo momento.
  \end{itemize}

  \item \textbf {Mano de Obra}
  \begin{itemize}
    \item Para la creacion de la empresa, se requiere de personal capacitado y
            competente acorde a los actuales estandares del desarrollo de sistemas
            de computacion para tener que evitar una previa capacitacion.
  \end{itemize}

  \item \textbf {Insumos}
  \begin{itemize}
    \item Se requiere de un buen aprovisionamiento de monitores de alta gama,
            ya que con ellos, los programadores pueden tener mas herramientas
            a la vista y agilizar su desarrollo tanto como su desempeño a la
            hora de analizar nuevos requerimientos y llevar a cabo las tareas
            necesarias de investigacion, desarrollo y pruebas del producto.
    \item Se requiere de un buen aprovisionamiento de teclados ergonomicos
            ya que los teclados comunes conducen a una postura poco saludable
            para todo aquel que pase mucho tiempo frente a una computadora.
            El desarrollador es la materia prima de la empresa y el teclado
            es su principal herramienta para el ingreso de sus ideas.
    \item Se requiere de un buen aprovisionamiento de mouse \(punteros de ordenaror\)
            ya que algunas aplicaciones, ya sean utilizadas en internet o de
            escritorio, requieren la accion de presionar sobre botones en la
            pantalla.
      \item Se requiere de gabinetes \(Caja rectangular donde se unen componentes
              alimentados por cables de electricidad unidos a una placa madre
              que ejecutan las tareas internas del ordenador\) armados
              con hardware \(Componentes individuales del ordenador\) capaz
              de procesar tareas complejas a una velocidad rapida y eficiente.
    \item Se requiere de un buen aprovisionamiento de cables de red ya que
            sin ellos los equipos y servidores que requieran de una conectividad
            fluida entre ellos y a Internet, dependerian de una conexion por
            red inalambrica, inestable y mucho mas lenta.
    \item Se requiere de un buen aprovisionamiento de modems \( Dispositivo
            que facilita la conexion entre un ordenador e Internet \) o
            Routers \( Dispositivo que facilita la conexion entre un
            ordenador e Internet u otros dispositivos conectados al mismo
            Router \) Wifi para conectividad de red sin cable.
    \item Se requiere de un Expansores de Señal de Wifi
  \end{itemize}
\end{itemize}

\subsection{Localizacion}
\begin{itemize}
  \item \textbf {Costo de transporte}
  \begin{itemize}
    \item No habria costo, porque cada uno podria trabajar
            desde su domicilio, siempre y cuando pueda cumplir
            las condiciones de los Materiales Necesarios.
  \end{itemize}

  \item \textbf {Comunicaciones}
  \begin{itemize}
    \item Todo tipo de comunicacion se realizaria a traves de internet
            compartiendo codigo del proyecto en un repositorio privado.
  \end{itemize}
\end{itemize}


\subsection{Requerimientos Tecnicos}
\begin{itemize}
  \item \textbf {Tecnologias}
  \begin{itemize}
    \item Librerias de libre uso y distribucion para el manejo de ventanas
          y herramientas varias (Botones ejecutables - Barras de desplazamiento )
          en el sistema operativo objetivo.
  \end{itemize}

  \item \textbf {Estructura Organizacional}
  \begin{itemize}
    \item Lider de Proyecto
    \item Administrador de Proyecto
    \item Desarrolladores
  \end{itemize}

  \item \textbf {Mapa de Proceso}
  \begin{itemize}
    \item Entrevista con el Cliente
    \item Relevamiento de Informacion a tener en cuenta
    \item Preparacion de Demostrativo
    \item Se plantean 2 semanas de continuas entregas de
            pequeñas caracteristicas funcionales a evaluacion
            del cliente para una mayor frecuencia de respuesta
            sobre el producto.

    \item Preparacion de etapa Pre-Produccion. Para ello se genera un ambiente
            en servidores maqueta para simular principales
            comportamientos del sistema.

    \item Preparacion de Puesta a Produccion del sistema en servidores del cliente
  \end{itemize}
\end{itemize}


\end{document}
