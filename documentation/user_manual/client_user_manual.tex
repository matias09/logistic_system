\documentclass[
10pt, % Main document font size
a4paper, % Paper type, use 'letterpaper' for US Letter paper
oneside, % One page layout (no page indentation)
%twoside, % Two page layout (page indentation for binding and different headers)
headinclude,footinclude, % Extra spacing for the header and footer
BCOR5mm, % Binding correction
]{scrartcl}

\usepackage{graphicx}
\usepackage{tikz-er2}
\usepackage[colorlinks, urlcolor=blue, linkcolor=black]{hyperref}

\title{Manual de Usuario - Sistema de Logistica}
\author{Matias M. Marceca}
\date{}

\begin{document}

\maketitle
\tableofcontents
\pagebreak

\section{Modulo Clientes}
\subsection{Vista de Busqueda}

 \begin{flushleft}
     \includegraphics[width=14cm, keepaspectratio]{clients_search_view_with_results_with_pic.jpg}
 \end{flushleft}

 \begin{quotation}
   En esta vista se listan los detalles de/los cliente/s buscados por "Nombre".
 \end{quotation}
 \begin{enumerated}
   \item  Elemento de insercion de caracteres alfanumericos
     ( letras y numeros ) para encontrar Clientes.
   \item  Seccion que muestra el listado de Cliente/s encontrados
      o Vacio en caso de no encontrar ninguno.

   \begin{enumerated}
     \item Se muestra el ID de Referencia del Cliente registrado.
     \item Se muestra el Nombre del Cliente registrado.
     \item Se muestra la Direccion Completa del Cliente registrado.
    \end{enumerated}

   \item Se Utiliza el Boton "Buscar" para ejecutar la Busqueda de Clientes.
   \item Se Utiliza el Boton "Nuevo" para ejecutar la Creacion de un Cliente.
  \end{enumerated}

\pagebreak

 \begin{flushleft}
     \includegraphics[width=10cm, keepaspectratio]{clients_new_view_with_pic.jpg}
 \end{flushleft}

 \begin{quotation}
    En esta vista se administra la Insercion de un Nuevo Cliente al sistema.
    Se cargan los detalles del cliente mas los datos de direccion seleccionando
    la casilla blanca a la derecha de cada propiedad del Cliente.
 \end{quotation}

 \begin{enumerated}
   \item  Elemento de insercion del Nombre.
   \item  Elemento de insercion del Telefono.
   \item  Elemento de insercion del Celular.
   \item  Elemento de insercion del E-Mail.
   \item  Elemento de insercion de la Provincia seleccionando
      del Menu a la derecha.
   \item  Elemento de insercion de la Calle.
   \item  Elemento de insercion de la Numeracion de la calle donde esta registrado.
   \item  Elemento de insercion del Codigo Postal.
   \item  Area de Notificacion en caso de Errores por validacion de Datos.
   \item  Boton que ejecuta el guardado persistente del Cliente.
  \end{enumerated}

\pagebreak

 \begin{flushleft}
     \includegraphics[width=10cm, keepaspectratio]{clients_edit_view_with_pic.jpg}
 \end{flushleft}

 \begin{quotation}
    En esta vista se administra la Modificacion de un Cliente ya existente
    en el sistema. Permite la modificacion de los detalles del cliente mas los
    datos de direccion seleccionando la casilla blanca a la derecha de cada
    propiedad del Cliente.
    Al mismo tiempo da la posibilidad de eliminar un Cliente "Siempre y cuando
    no tenga Viajes en Curso".
 \end{quotation}

 \begin{enumerated}
   \item  Elemento de insercion del Nombre.
   \item  Elemento de insercion del Telefono.
   \item  Elemento de insercion del Celular.
   \item  Elemento de insercion del E-Mail.
   \item  Elemento de insercion de la Provincia seleccionando
      del Menu a la derecha.
   \item  Elemento de insercion de la Calle.
   \item  Elemento de insercion de la Numeracion de la calle donde esta registrado.
   \item  Elemento de insercion del Codigo Postal.
   \item  Area de Notificacion en caso de Errores por validacion de Datos.
   \item  Boton que ejecuta el guardado persistente del Cliente.
  \end{enumerated}

\end{document}
